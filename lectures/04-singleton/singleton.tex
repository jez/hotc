\documentclass{article}
\usepackage[a5paper, margin=0.5in]{geometry}
\usepackage{amsmath, amssymb, latexsym}
\usepackage{mathpartir, proof}
\usepackage[linktoc=all]{hyperref}

\title{The Singleton Kind Calculus}
\date{February 2, 2017}

\newcommand{\type}{\ensuremath{\mathtt{type}}}
\newcommand{\kind}{\ensuremath{\mathtt{kind}}}

\begin{document}
\maketitle

\begin{abstract}
  In this section we develop the singleton kind calculus. A singleton kind
  $S(c)$ is the kind of all constructors that are equivalent to $c$. The
  addition of these new kinds will be useful to explain module signatures later
  on.
\end{abstract}

\section{Syntax}

The singleton kind calculus is built on top of souped-up $F_\omega$.

\[
  \begin{array}{lcl}
    k & ::= & \type \mid k \to k \mid k * k \mid S(c) \mid \Pi \alpha : k. \; k \mid \Sigma \alpha : k. \; k\\
    c & ::= & \alpha \mid c \to c \mid \forall \alpha : k. c
        \mid \lambda \alpha : k. c \mid c\ c \mid \langle c, c \rangle \mid \pi_1 \ c \mid \pi_2 \ c \\
    e & ::= & x \mid \lambda x : c. e \mid e\ e \mid
            \Lambda \alpha : k. e \mid e[c]\\
    \Gamma & ::= & \cdot \mid \Gamma, x : c \mid \Gamma, \alpha : k
  \end{array}
\]

\section{Motivation}

Consider the following ML signature.

\begin{verbatim}
sig
    type t
    type 'a u
    type ('a, 'b) v
    type w = int
end
\end{verbatim}

The first three types can be assigned kinds in $F_\omega$ in a straight forward
way.

\[
  \begin{array}{l}
    \verb|t| : \type \\
    \verb|u| : \type \to \type \\
    \verb|v| : \type \times \type \to \type
  \end{array}
\]

But how do we kind \verb|w|? Remember, \verb|int| is not a kind, so it doesn't
make sense to say \verb|w| : \verb|int|. But it's not quite right to say
$\verb|w| : \type$ either, because \verb|w| cannot stand for arbitrary types. We
therefore write $\verb|w| : S(\verb|int|)$: $S(\verb|int|)$ is the kind
containing exactly $\verb|int|$ and all those types equivalent to $\verb|int|$,
such as $(\lambda \alpha : \type. \; \alpha) \; \verb|int|$. The other new kind
constructs, $\Pi \alpha : k. \; k$ and $\Sigma \alpha : k. \; k$ (which are
called dependent function spaces and dependent sums respectively), exist to
solve the analogous problem for kinding assignments to polymorphic types in
signatures. Specifically, consider this example:

\begin{verbatim}
sig
    type 'a t = 'a list
end
\end{verbatim}

We can say that

\[
  \begin{array}{l}
    \verb|t| : \Pi \alpha : \type. \; \mathcal S(\alpha)
  \end{array}
\]

\(\Sigma\) kinds are used to kind types that are assigned to abstract types. In
the signature

\begin{verbatim}
    type t
    type s = t
\end{verbatim}

we say that

\[
  \begin{array}{l}
    \verb|t| : \type \\
    \verb|s| : \Sigma \alpha : \type. \; \mathcal S(\alpha)
  \end{array}
\]


\section{Definitions}

In this section the following judgements will be defined.

\[
    \begin{array}{ll}
        \text{Judgement} & \text{Description} \\
        \hline
        \Gamma \vdash k : \kind & k \text{ is a kind} \\
        \Gamma \vdash k \equiv k' : \kind & \text{kind equivalence} \\
        \Gamma \vdash k \leq k' & \text{subkinding} \\
        \Gamma \vdash c : k & \text{$c$ has kind $k$} \\
        \Gamma \vdash c \equiv c' : k & \text{constructor equivalence} \\
        \Gamma \vdash e : \tau & \text{$e$ has type $\tau$}
    \end{array}
\]
A complete list would also include the judgement $\Gamma \vdash \tau : \type$
but these rules are exactly the same as in $F_\omega$ so we will omit them.

\subsection{Well-Formed Kinds}

\begin{mathpar}
  \inferrule[3a]{\quad}{\Gamma \vdash \tau : \kind} \and
  \inferrule[3b]{
    \Gamma \vdash c : \tau
  }{
    \Gamma \vdash S(c) : \kind
  } \and

  \inferrule[3c]{
    \Gamma \vdash k_1 : \kind \quad \Gamma, k_1 : \kind \vdash k_2 : \kind
  }{
    \Gamma \vdash \Pi \alpha : k_1. \ k_2 : \kind
  } \and

  \inferrule[3d]{
    \Gamma \vdash k_1 : \kind \quad \Gamma, k_1 : \kind \vdash k_2 : \kind
  }{
    \Gamma \vdash \Sigma \alpha : k_1. \ k_2 : \kind
  }
\end{mathpar}

\subsection{Definitional Equality of Kinds}

\begin{mathpar}
  \inferrule[3e]{
    \Gamma \vdash k : \kind
  }{
    \Gamma \vdash k \equiv k : \kind
  } \and

  \inferrule[3f]{
    \Gamma \vdash k_1 \equiv k_2 : \kind
  }{
    \Gamma \vdash k_2 \equiv k_1 : \kind
  } \and

  \inferrule[3g]{
    \Gamma \vdash k_1 \equiv k_2 : \kind \quad \Gamma \vdash k_2 \equiv k_3 : \kind
  }{
    \Gamma \vdash k_1 \equiv k_3 : \kind
  } \and

  \inferrule[3h]{
    \Gamma \vdash c \equiv c' : \type
  }{
    \Gamma \vdash S(c) \equiv S(c') : \kind
  } \and

  \inferrule[3i]{
    \Gamma \vdash k_1 \equiv k_1' : \kind \quad \Gamma, \alpha : k_1 \vdash k_2 \equiv k_2' : \kind
  }{
    \Gamma \vdash \Pi \alpha : k_1. \; k_2 \equiv \Pi \alpha : k_1'. \; k_2 : \kind
  } \and

  \inferrule[3j]{
    \Gamma \vdash k_1 \equiv k_1' : \kind \quad \Gamma, \alpha : k_1 \vdash k_2 \equiv k_2' : \kind
  }{
    \Gamma \vdash \Sigma \alpha : k_1. \; k_2 \equiv \Sigma \alpha : k_1'. \; k_2 : \kind
  }
\end{mathpar}

\subsection{Kind Membership}

This judgement defines which constructors belong to a given kind.

\begin{mathpar}
  \inferrule[3k]{
    \alpha : k \in \Gamma
  }{
    \Gamma \vdash \alpha : k
  } \and

  \inferrule[3l]{
    \Gamma \vdash c_1 : \type \quad \Gamma \vdash c_2 : \type
  }{
    \Gamma \vdash c_1 \to c_2 : \type
  } \and

  \inferrule[3m]{
    \Gamma \vdash k : \kind \quad \Gamma, \alpha : k \vdash c : \type
  }{
    \Gamma \vdash \forall \alpha : k. \; c : \type
  } \and

  \inferrule[3n]{
    \Gamma \vdash k_1 : \kind \quad \Gamma, \alpha : k_1 \vdash c : k_2
  }{
    \Gamma \vdash \lambda \alpha : k_1. \; c : \Pi \alpha : k_1. \; k_2
  } \and

  \inferrule[3o]{
    \Gamma \vdash c_1 : \Pi \alpha : k. \; k' \quad \Gamma \vdash c_2 : k
  }{
    \Gamma \vdash c_1 \; c_2 : [c_2/\alpha]k'
  } \and

  \inferrule[3p]{
    \Gamma \vdash c_1 : k_1 \quad \Gamma \vdash c_2 : [c_1/\alpha]k_2 \quad \Gamma, \alpha : k_1 \vdash k_2 : \kind
  }{
    \Gamma \vdash \langle c_1, c_2 \rangle : \Sigma \alpha : k_1. \; k_2
  } \and

  \inferrule[3q]{
    \Gamma \vdash c : \Sigma \alpha : k_1. \; k_2
  }{
    \Gamma \vdash \pi_1 \ c : k_1
  } \and

  \inferrule[3r]{
    \Gamma \vdash c : \Sigma \alpha : k_1. \; k_2
  }{
    \Gamma \vdash \pi_2 \ c : [\pi_1 c/\alpha]k_2
  } \and

  \inferrule[3s]{
    \Gamma \vdash c : \type
  }{
    \Gamma \vdash c : S(c)
  }
\end{mathpar}

Notice that even though $\Sigma \alpha : k_1. \ k_2$ is called a dependent sum,
it behaves like a product. To avoid the obvious naming confusion here, we will
try our best to call $\Sigma \alpha : k_1. \ k_2$ a "dependent sum" and $\Pi
\alpha : k_1. \ k_2$ a "dependent function space" or sometimes just a "dependent
function" because the former is a bit of a mouthful.

\subsection{Subkinding}

\begin{mathpar}
    \inferrule[3t]{
      \Gamma \vdash k : \kind
    }{
      \Gamma \vdash k \leq k
    } \and

    \inferrule[3u]{
      \Gamma \vdash k_1 \leq k_2 \quad \Gamma \vdash k_2 \leq k_3
    }{
      \Gamma \vdash k_1 \leq k_3
    } \and

    \inferrule[3v]{
      \Gamma \vdash c : \type
    }{
      \Gamma \vdash S(c) \leq \type
    } \and

    \inferrule[3w]{
      \Gamma \vdash c \equiv c' : \type
    }{
      \Gamma \vdash S(c) \leq S(c') : \type
    } \and

    \inferrule[3x]{
      \Gamma \vdash k_1' \leq k_1 \quad \Gamma \alpha : k_1 \vdash k_2 : \kind \quad \Gamma, \alpha : k_1' \vdash k_2 \leq k_2'
    }{
      \Gamma \vdash \Pi \alpha : k_1. \; k_2 \leq \Pi \alpha : k_1'. \; k_2'
    } \and

    \inferrule[3y]{
      \Gamma \vdash k_1 \leq k_1' \quad \Gamma, \alpha : k_1 \vdash k_2 \leq k_2' \quad \Gamma, \alpha : k_1' \vdash k_2' : \kind
    }{
      \Gamma \vdash \Sigma \alpha : k_1. \; k_2 \leq \Sigma \alpha : k_1'. \; k_2'
    } \and

    \inferrule[3z]{
      \Gamma \vdash c : k \quad \Gamma \vdash k \leq k'
    }{
      \Gamma \vdash c : k'
    } \and

    \inferrule[3aa]{
      \Gamma \vdash k \equiv k'
    }{
      \Gamma \vdash k \leq k'
    }
\end{mathpar}

\subsection{Definitional Equality of Constructors}

\begin{mathpar}
    \inferrule[3ab]{
      \Gamma \vdash c : k
    }{
      \Gamma \vdash c \equiv c : k
    } \and

    \inferrule[3ac]{
      \Gamma \vdash c_1 \equiv c_2 : k
    }{
      \Gamma \vdash c_2 \equiv c_1 : k
    } \and

    \inferrule[3ad]{
      \Gamma \vdash c_2 : k \quad \Gamma, \alpha : k \vdash c_1 : k'
    }{
      \Gamma \vdash (\lambda \alpha : k. \; c_1) \; c_2 \equiv [c_2/\alpha]c_1 : [c_2/\alpha]k'
    } \and

    \inferrule[3ae]{
      \Gamma \vdash c_1 : k_1 \quad \Gamma \vdash c_2 : k_2
    }{
      \Gamma \vdash \pi_1 \langle c_1, c_2 \rangle \equiv c_1
    } \and

    \inferrule[3af]{
      \Gamma \vdash c_1 : k_1 \quad \Gamma \vdash c_2 : k_2
    }{
      \Gamma \vdash \pi_2\langle c_1, c_2 \rangle \equiv c_2
    } \and

    \inferrule[3ag]{
      \Gamma \vdash c_1 \equiv c_2 : k \quad \Gamma \vdash k \leq k'
    }{
      \Gamma \vdash c_1 \equiv c_2 : k'
    } \and

    \inferrule[3ah]{
      \Gamma \vdash c \equiv c' : \type
    }{
      \Gamma \vdash c \equiv c' : S(c)
    } \and

    \inferrule[3ai]{
      \Gamma \vdash c : S(c')
    }{
      \Gamma \vdash c \equiv c' : \type
    }

\end{mathpar}

\begin{mathpar}
    \inferrule[3aj]{
      \Gamma \vdash k_1 \equiv k_2 : \kind \quad \Gamma, \alpha : k_1 \vdash c \equiv c' : k_2
    }{
      \Gamma \vdash \lambda \alpha : k_1. \; c \equiv \lambda \alpha : k_1'. \; c' : \Pi \alpha : k_1. \; k_2
    } \and

    \inferrule[3ak]{
      \Gamma \vdash c \equiv c' : \Pi \alpha : k. \; k' \quad \Gamma \vdash c_2 \equiv c_2' : k
    }{
      \Gamma \vdash c_1 \; c_2 \equiv c_1' \; c_2' : [c_2/\alpha]k'
    } \and

    \inferrule[3al]{
      \Gamma \vdash c_1 \equiv c_1' : k_1 \quad \Gamma \vdash c_2 \equiv c_2' : [c_1/\alpha]k_2 \quad \Gamma, \alpha : k_1 \vdash k_2 : \kind
    }{
      \Gamma \vdash \langle c_1, c_2 \rangle \equiv \langle c_1', c_2' \rangle : \Sigma \alpha : k_1. \; k_2
    } \and

    \inferrule[3am]{
      \Gamma \vdash c \equiv c' : \Sigma \alpha : k_1. \; k_2
    }{
      \Gamma \vdash \pi_1 \; c \equiv \pi_1 \; c' : k_1
    } \and

    \inferrule[3an]{
      \Gamma \vdash c \equiv c' : \Sigma \alpha : k_1. \; k_2
    }{
      \Gamma \vdash \pi_2 \; c \equiv \pi_2 \; c' : [\pi_1 \; c/\alpha] k_2
    } \and

    \inferrule[3an]{
      \Gamma \vdash c_1 \equiv c_1' : \type \quad \Gamma \vdash c_2 \equiv c_2' : \type
    }{
      \Gamma \vdash c_1 \to c_2 \equiv c_1' \to c_2' : \type
    } \and

    \inferrule[3ao]{
      \Gamma \vdash k \equiv k' : \kind \quad \Gamma, \alpha : k \vdash c \equiv c' : \type
    }{
      \Gamma \vdash \forall \alpha : k. \; c \equiv \forall \alpha : k'. c' : \type
    } \and

    \inferrule[3ap]{
      \Gamma, \alpha : k_1 \vdash c \; \alpha \equiv c' \; \alpha : k_2 \quad \Gamma \vdash c : \Pi \alpha : k_1. \; k_2 \Gamma \vdash c' : \Pi \alpha : k_1. \; k_2''
    }{
      \Gamma \vdash c \equiv c' : \Pi \alpha : k_1. \; k_2
    } \and

    \inferrule[3aq]{
      \Gamma \vdash \pi_1 \; c \equiv \pi_1 \; c' \quad \Gamma, \alpha : k_1 \vdash k_2 : \kind \quad \Gamma \vdash \pi_2 \; c \equiv \pi_2 \; c' : [\pi_1 \; c/\alpha]k_2
    }{
      \Gamma \vdash c \equiv c' : \Sigma \alpha : k_1. \; k_2
    }
\end{mathpar}

\end{document}
