\documentclass{article}

\usepackage[a5paper, margin=0.5in]{geometry}
\usepackage{mathpartir}
\usepackage{amsmath}
\usepackage{amssymb}
\usepackage{amsfonts}
\usepackage{latexsym}
\usepackage[linktoc=all]{hyperref}

\newcommand{\kwd}[1]{\ensuremath{\mathtt{#1}}}
\newcommand{\intt}{\ensuremath{\mathtt{int}}}
\newcommand{\vect}[1]{\langle #1 \rangle}

\title{Closure conversion}
\date{February 28, 2017}

\begin{document}
\maketitle

\begin{abstract}
  To this point we've been working with functions that have free variables, like
  $\lambda x. \; f \; g \; x$. In theory, instantiation of free variables is
  implemented by substitution, but hardware does not support this operation.
  Thus we now define a language, IL-Closure, in which terms explicitly carry
  environments defining their free variables, ``closing'' over them. Then, we
  show how to translate IL-CPS into IL-Closure.
\end{abstract}

\section{A brief note on implementation}

Closures are expensive, and a robust implementation should avoid creating them
whenever possible. So, though this translation pass is needed to take care of
higher-order usage of functions, ``known'' defined at the top-level should not
be converted into closures.

\section{Examples}

Consider the term
\[
  \lambda x : \intt. \; x + y + z
\]
Translating this to a closure, we get

\begin{align*}
  &\vect{\lambda x : \intt. \; \lambda env : \intt \times \intt. \; \\
  & \; \; \; \kwd{let} \; y = \pi_0 \; env \; \kwd{in} \\
  & \; \; \; \kwd{let} \; z = \pi_1 \; env \; \kwd{in} \\
  & \; \; \; x + y + z \\
  &, \vect{y, z}}
\end{align*}

As a consequence, for function application
\[
  e \; 5
\]
we need to translate as well.

\begin{align*}
  &\kwd{let} \; f = \pi_0 \; e \; \kwd{in} \\
  &\kwd{let} \; env = \pi_1 \; e \; \kwd{in} \\
  &f \; env \; 5
\end{align*}

But in order to get type-directed translation to go through smoothly, we will
need to be somewhat clever. Consider the following ``problem'' case.

\begin{align*}
  \kwd{if} \; b \; \kwd{then} \; (\lambda x : \intt. \; x + y) \; \kwd{else} \; (\lambda x : \intt. \; x + y + z)
\end{align*}

We want to be able to give some $\tau_{env}$ such that
\[
  \overline{\intt \to \intt} \; = (\intt \to \tau_{env} \to \intt) \times \tau_{env}
\]

But the above example shows that some terms of type $\intt \to \intt$ do not
admit one correct choice $\tau_{env}$ type. The solution is to make the type
existentially quantified.
\[
  \overline{\intt \to \intt} \; = \exists \alpha_{env} : \tau_{env}. \; (\intt \to \alpha_{env} \to \intt) \times \alpha_{env}
\]

\section{Syntax and judgements}

The syntax for IL-Closure is the same as the syntax for IL-CPS. The two
principal typing judgements for this language are

\[
  \begin{array}{l}
    \Delta ; \Gamma \vdash e : 0 \\
    \Delta ; \Gamma \vdash v : \tau
  \end{array}
\]
The rule of interest is as follows.

\begin{mathpar}
  \inferrule[3a]{\Delta \vdash \tau \; \kwd{type} \and \Delta; \cdot, x : \tau \vdash e : 0}{\Delta; \Gamma \vdash \lambda x : \tau. \; e : \neg \tau} \and
\end{mathpar}

\section{Translation}

Type translation is straightforward mapping through, except at arrow types.

\begin{align*}
    \overline{\alpha} &= \alpha \\
    &... \\
    \overline{\tau_1 \to \tau_2} &= \exists \tau_{env}. \; (\tau_1 \to \tau_{env} \to \tau_2) \times \tau_{env}
\end{align*}

The rules of interest are as follows.

\begin{mathpar}
    \inferrule[4a]{\Delta \vdash \tau : \kwd{type} \and \Delta ; \Gamma, x : \tau \vdash e : 0 \leadsto \overline{e} \and \Gamma = x_1 : \tau_1, \dots, x_n : \tau_n}{\Delta ; \Gamma \vdash \lambda x : \tau. \; e : \neg \tau \leadsto
    \begin{tabular}{l}
        $\kwd{pack} \; [ \overline{\tau_1} \times \cdots \times \overline{\tau_n},$ \\
        $ \; \; \; \langle ( \lambda y : \overline{\tau} \times (\overline{\tau_1} \times \cdots \times \overline{\tau_n}).$ \\
        $ \; \; \; \; \; \; \; \kwd{let} \; x = \pi_1 \; y \; \kwd{in}$ \\
        $ \; \; \; \; \; \; \; \kwd{let} \; env = \pi_2 \; y \; \kwd{in}$ \\
        $ \; \; \; \; \; \; \; \kwd{let} \; x_1 = \pi_1 \; env \; \kwd{in}$ \\
        $ \; \; \; \; \; \; \; \cdots$ \\
        $ \; \; \; \; \; \; \; \kwd{let} \; x_n = \pi_n \; env \; \kwd{in}$ \\
        $ \; \; \; \; \; \; \; \overline{e}),$ \\
        $ \; \; \; \langle x_1, \cdots, x_n \rangle \rangle]$ \\
        $\kwd{as} \; \exists \alpha_{env} : \kwd{type}. \; \neg(\overline{\tau} \times \alpha_{env}) \times \alpha_{env}$
    \end{tabular}} \and
    \inferrule[4b]{\Delta; \Gamma \vdash v_1 : \neg \tau \leadsto \overline{v_1} \and \Delta; \Gamma \vdash v_2 : \tau \leadsto \overline{v_2}}{\Delta; \Gamma \vdash v_1 \; v_2 : 0 \leadsto
     \begin{tabular}{l}
     $\kwd{unpack} \; [\alpha_{env}, x] = \overline{v_1} \; \kwd{in}$ \\
     $\; \; \; \kwd{let} \; f = \pi_1 \; x \; \kwd{in}$ \\
     $\; \; \; \kwd{let} \; env = \pi_2 \; x \; \kwd{in}$ \\
     $\; \; \; f \vect{\overline{v_2}, env}$
     \end{tabular}
    }
\end{mathpar}

\end{document}
